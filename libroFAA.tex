\documentclass{book}
\usepackage[spanish]{babel}
\usepackage{mathtools}
\title{El libro de FAA}	
\date{}
\author{Jesús García Gutiérrez}
\begin{document}
	\maketitle{}
	\chapter{Relación de recurrencia}
	\section{Ejercicio del 02/04/2025}
	%Source: https://tex.stackexchange.com/questions/365953/how-do-i-define-a-piecewise-function-in-latex
	\paragraph{Hallar la eficiencia de un algoritmo dado. Supongamos que ya nos dan la función a trozos del T(n).}
	\[ T(n) = 	\begin{cases} 
					1 & n\leq 1 \\
					T(\frac{n}{2}) + 2 & n > 1 
				\end{cases}
       \]
       \paragraph{Primero aislamos la relación de recurrencia. Las relaciones de recurrencia tienen en la parte derecha de la igualdad una o más T, a diferencia de los casos base, que no tienen ninguna T en dicha parte.}
      \begin{equation}
	      T(n) = T(\frac{n}{2}) + 2  
      \end{equation}
      \paragraph{En este caso, podemos aplicar tanto el método de expansión de recurrencia como el teorema maestro. Esto es así porque solo hay una T en la parte derecha de la igualdad. Si hubiera más de una T en dicha parte, no podríamos aplicar ninguno de estos métodos.}
      \subsection{Método de expansión de recurrencia}
      \paragraph{El método de expansión de recurrencia consiste en hallar el tiempo para problemas cada vez más pequeños, y calcular el T(n) utilizando cada uno de estos hallazgos.}
      \paragraph{Si nos fijamos en la ecuación 1.1, vemos $T(\frac{n}{2})$. ¡Vamos a hallarlo usando T(n)!}
	\begin{equation}	
		T(\frac{n}{2}) = T(\frac{(\frac{n}{2})}{2}) + 2 \nonumber
	\end{equation}	
	\begin{equation}	
		T(\frac{n}{2}) = T(\frac{n}{4}) + 2 \nonumber 
	\end{equation}	
	\paragraph{Listo, hemos desarrollado $T(\frac{n}{2})$. Ahora podemos sustituir la expresión que hemos hallado en la ecuación 1.1.}
	\begin{equation}	
		T(n) = (T(\frac{n}{4}) + 2) + 2 \nonumber 
	\end{equation}	
	\begin{equation}	
		T(n) = T(\frac{n}{4}) + 4 
	\end{equation}	
	\paragraph{Perfecto, la ecuación 1.2 es una nueva fórmula para el T(n).}
	\paragraph{Ahora solo tenemos que seguir el mismo procedimiento que antes, hasta que veamos un patrón claro y podamos escribirlo en una ecuación. ¡Vamos con el $T(\frac{n}{4})$ que aparece en la ecuación 1.2!}
	\begin{equation}	
		T(\frac{n}{4}) = T(\frac{(\frac{n}{4})}{2}) + 2 \nonumber 
	\end{equation}
	\begin{equation}
		T(\frac{n}{4}) = T(\frac{n}{8}) + 2 \nonumber 
	\end{equation}
	\paragraph{Listo, sustituimos en la ecuación 1.2.}
	\begin{equation}
		T(n) = (T(\frac{n}{8}) + 2) + 4 \nonumber  
	\end{equation}
	\begin{equation}
		T(n) = T(\frac{n}{8}) + 6 
	\end{equation}
	\paragraph{Ya lo vas pillando, ¡a por el $T(\frac{n}{8})$!}
	\begin{equation}
		T(\frac{n}{8}) = T(\frac{(\frac{n}{8})}{2}) + 2 \nonumber 
	\end{equation}
	\begin{equation}
		T(\frac{n}{8}) = T(\frac{n}{16}) + 2 \nonumber 
	\end{equation}	
	\paragraph{Sustituimos $T(\frac{n}{8})$ en la ecuación 1.3.}
	\begin{equation}
		T(n) = (T(\frac{n}{16}) + 2) + 6 \nonumber  
	\end{equation}	
	\begin{equation}
		T(n) = T(\frac{n}{16}) + 8 
	\end{equation}	
	\paragraph{Ya hemos hallado el T(n) utilizando varios subproblemas. Si nos fijamos en las ecuaciones 1.1, 1.2, 1.3 y 1.4, ¿veis el patrón? Vamos a escribirlo utilizando una variable llamada i.}
	\begin{equation}
		T(n) = T(\frac{n}{2^i}) + 2 \cdot i 
	\end{equation}
	\paragraph{Estupendo, la ecuación 1.5 es muy bonita y parece que ya somos informáticos teóricos cum laude, pero ocurre una desgracia: no está solo en función de n, también está en función de i. Tenemos que transformar i en n de algún modo.}
	\paragraph{En la ecuación 1.5 fijémonos en la expresión $T(\frac{n}{2^i})$. Solo tenemos que igualar el paréntesis al valor de n que sea caso base. Si volvemos a la función a trozos, vemos que el caso base tiene $n=1$, y por tanto podemos escribir lo siguiente.}
	\begin{equation}
		\frac{n}{2^i} = 1 \nonumber 
	\end{equation}
	\paragraph{Podríamos pensar que $n=0$ es también un caso base, pero no trabajaremos nunca con problemas de tamaño igual a cero. Por eso ponemos 1 y no 0 en la igualdad. Seguimos desarrollando y aplicando la idea de logaritmo.}
	\begin{equation}
	   	n = 2^i \nonumber 
	\end{equation}
	\begin{equation}
		\log_{2}(n) = i \nonumber 
	\end{equation}
	\begin{equation}
	   	i = \log_{2}(n) \nonumber 
	\end{equation}
	\paragraph{Listo, tenemos i en función de n. Sustituimos la i en la ecuación 1.5.}
	%Source: https://tex.stackexchange.com/questions/475687/how-to-write-pow-math
	\begin{equation}
		T(n) = T(\frac{n}{2^\mathrm{\log_{2}(n)}}) + 2 \cdot \log_{2}(n)  \nonumber 
	\end{equation}
	\paragraph{El denominador se puede seguir simplificando matemáticamente.}
	\begin{equation}
		2^\mathrm{\log_{2}(n)} = n^\mathrm{\log_{2}(2)} \nonumber 
	\end{equation}
	\begin{equation}
		2^\mathrm{\log_{2}(n)} = n^1 \nonumber 
	\end{equation}
	\paragraph{Maravilloso, ahora el denominador es simplemente n. Volvemos a donde estábamos antes.}
	\begin{equation}
		T(n) = T(\frac{n}{n}) + 2 \cdot \log_{2}(n) \nonumber 
	\end{equation}
	\begin{equation}
		T(n) = T(1) + 2 \cdot \log_{2}(n) \nonumber 
	\end{equation}
	\paragraph{Sustituimos T(1), sabiendo por la función a trozos que $T(1) = 1$}
	\begin{equation}
		T(n) = 1 + 2 \cdot \log_{2}(n) \nonumber 
	\end{equation}
	\begin{equation}
		T(n) = 2 \cdot \log_{2}(n) + 1 
	\end{equation}
	\paragraph{Listo, qué bonita nos ha quedado la fórmula 1.6. Esta nos da el número de operaciones elementales dada una n, y no tenemos que hacer tiempos de otros subproblemas como ocurría con la relación de recurrencia originalmente.}
	\paragraph{También podemos hablar ya del orden de complejidad. Podemos ver que...}
	\begin{equation}
		T(n) \in O(\log_{2}(n)) \nonumber
	\end{equation}		
\end{document}